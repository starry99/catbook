\documentclass[10pt, a4paper]{article}
\usepackage[top=3cm, bottom=4cm, left=3.5cm, right=3.5cm]{geometry}
\usepackage{amsmath,amsthm,amsfonts,amssymb,amscd, fancyhdr, color, comment, graphicx, environ, pifont}
\usepackage{float}
\usepackage{mathrsfs}
\usepackage[math-style=ISO]{unicode-math}
\usepackage[framemethod=TikZ]{mdframed}
\usepackage{enumerate}
\usepackage[shortlabels]{enumitem}
\usepackage{fancyhdr}
\usepackage{indentfirst}
\usepackage{listings}
\usepackage{sectsty}
\usepackage{thmtools}
\usepackage{shadethm}
\usepackage{hyperref}
\usepackage{setspace}
\usepackage[linguistics]{forest}
\hypersetup{
    colorlinks=true,
    linkcolor=blue,
    filecolor=magenta,      
    urlcolor=blue,
}
\mdfsetup{skipabove=\topskip,skipbelow=\topskip}
\mdfdefinestyle{theoremstyle}{%
linecolor=black,linewidth=1pt,%
frametitlerule=true,%
frametitlebackgroundcolor=gray!20,
innertopmargin=\topskip,
}
\mdtheorem[style=theoremstyle]{Problem}{Problem}
\newenvironment{Solution}{\textbf{Solution.}}

\definecolor{codegreen}{rgb}{0,0.6,0}
\definecolor{codegray}{rgb}{0.5,0.5,0.5}
\definecolor{codepurple}{rgb}{0.58,0,0.82}
\definecolor{backcolour}{rgb}{0.95,0.95,0.92}

\lstdefinestyle{mystyle}{
    backgroundcolor=\color{backcolour},   
    commentstyle=\color{codegreen},
    keywordstyle=\color{magenta},
    numberstyle=\tiny\color{codegray},
    stringstyle=\color{codepurple},
    basicstyle=\ttfamily\footnotesize,
    breakatwhitespace=false,         
    breaklines=true,                 
    captionpos=b,                    
    keepspaces=true,                 
    numbers=left,                    
    numbersep=5pt,                  
    showspaces=false,                
    showstringspaces=false,
    showtabs=false,                  
    tabsize=2
}

\lstset{style=mystyle}
\newcommand{\norm}[1]{\left\lVert#1\right\rVert}     
\newcommand\course{Analysis I}                            
\newcommand\hwnumber{MATH40002}                                  
\pagestyle{fancy}
\headheight 35pt
\lhead{\today}
\rhead{\includegraphics[width=2.5cm]{icl_logo.png}}
\lfoot{}
\pagenumbering{arabic}
\cfoot{\small\thepage}
\rfoot{}
\headsep 1.2em
\renewcommand{\baselinestretch}{1.25}
\renewcommand{\labelenumi}{\alph{enumi}}
\newcommand{\Z}{\mathbb Z}
\newcommand{\R}{\mathbb R}
\newcommand{\Q}{\mathbb Q}
\newcommand{\NN}{\mathbb N}
\newcommand{\PP}{\mathbb P}
\DeclareMathOperator{\Mod}{Mod} 
\renewcommand\lstlistingname{Algorithm}
\renewcommand\lstlistlistingname{Algorithms}
\def\lstlistingautorefname{Alg.}
\newtheorem*{theorem}{Theorem}
\newtheorem*{lemma}{Lemma}
\newtheorem{case}{Case}
\newcommand{\assign}{:=}
\newcommand{\infixiff}{\text{ iff }}
\newcommand{\nobracket}{}
\newcommand{\backassign}{=:}
\newcommand{\tmmathbf}[1]{\ensuremath{\boldsymbol{#1}}}
\newcommand{\tmop}[1]{\ensuremath{\operatorname{#1}}}
\newcommand{\tmtextbf}[1]{\text{{\bfseries{#1}}}}
\newcommand{\tmtextit}[1]{\text{{\itshape{#1}}}}

\newenvironment{itemizedot}{\begin{itemize} \renewcommand{\labelitemi}{$\bullet$}\renewcommand{\labelitemii}{$\bullet$}\renewcommand{\labelitemiii}{$\bullet$}\renewcommand{\labelitemiv}{$\bullet$}}{\end{itemize}}
\catcode`\<=\active \def<{
\fontencoding{T1}\selectfont\symbol{60}\fontencoding{\encodingdefault}}
\catcode`\>=\active \def>{
\fontencoding{T1}\selectfont\symbol{62}\fontencoding{\encodingdefault}}
\catcode`\<=\active \def<{
\fontencoding{T1}\selectfont\symbol{60}\fontencoding{\encodingdefault}}
\begin{document}

\begin{titlepage}
    \begin{center}
        \vspace*{3cm}
            
        \Huge
        \textbf{
        Coursework III}
            
            
        \vspace{1.5cm}
        \Large
            
        \textbf{
        CID number: Oh, no. This number series seeems to lost to some space!}% <-- author
        
            
        \vfill
        
MATH40002: Analysis I
        \vspace{1cm}
            
        \includegraphics[width=0.4\textwidth]{icl_logo.png}
        \\
        
        \Large
        
        \today
            
    \end{center}
\end{titlepage}


\newpage
\begin{Problem}
    Let $A$ and $B$ be non-empty subsets of $\mathbb{R}$ such that $a \leq b$ for all $a \in A$ and $b \in B$. Show that $\sup A \leq \inf B$ and that the equality holds if and only if for all $\varepsilon>0$, there are $a \in A$ and $b \in B$ such that $b-a<\varepsilon$.
\end{Problem}
\begin{Solution}
    \begin{proof}
    For the first part, "$\sup A \leq \inf B$" , we will prove this by contradiction. Let's assume that 
    \begin{equation}
    \sup A > \inf B
    \end{equation}
    Therefore, we can always find a number $c>0$, such that $\sup A = \inf B + c$, which is equivalent to $\inf B = \sup A - c$. By the definiton of $supreme$ and the proposition we learnt in the autumn term of Analysis I, we have
    \begin{equation}
        \exists \;a_0 \in A, \:\text{such that} \; a_0>\inf B.
    \end{equation}
    Now, by the definition given of the subsets  $A$ and $B$, we have
    $$
    \forall a \in A,\; \forall b \in B \Longrightarrow  a \leq b.
    $$
    Therefore, for every $a\in A$, $a$ is a lower bound of $B$. Hence, by the definiton of $infimum$, we have
    $$
    \forall a \in A,\; a\leq \inf B.
    $$
    which is a contradiction of $(2)$. Therefore, by the axiom of Trichotomy, we have 
    \begin{equation*}
        \sup A \leq \inf B.
    \end{equation*}
    For the second part, "$\sup A = \inf B$ $\Longleftrightarrow$ $\forall \varepsilon > 0,\: \exists a\in A,\: \exists b \in B,\: \text{such that}\: b-a<\varepsilon$" , we first prove the "$\Longrightarrow $" direction.\\
        "$\Longrightarrow $":\\
             We suppose that $\sup A = \inf B$ and let $\varepsilon>0$ be arbitrary. Then, we can choose a $a\in A$, such that
             \begin{equation}
                \sup A -\frac{\varepsilon}{2} < a \leq \sup A
             \end{equation}
             and $b\in B$, such that
             \begin{equation}
                \inf B \leq b < \inf B + \frac{\varepsilon}{2}
             \end{equation}
             This is possible since $\sup A$ and $\inf B$ are the least upper bound and the greatest lower bound of $A$ and $B$, respectively. Then, we combine $(3)$ and $(4)$, by the property of inequalities,
             \begin{equation}
                b-a< \inf B + \frac{\varepsilon}{2} - (\sup A - \frac{\varepsilon}{2}) = \inf B - \sup A + \varepsilon = \varepsilon
             \end{equation}
             since $\sup A = \inf B$, and we have proved the "$\Longrightarrow $" direction.\\
        "$\Longleftarrow $":\\
        For this, we will prove this by contradiction. We suppose that $\forall \varepsilon > 0,\: \exists a\in A,\: \exists b \in B,\: \text{such that}\: b-a<\varepsilon$. We have already proved that $\sup A \leq \inf B$, so we can only assume that $\sup A < \inf B$. Then, we set
        \begin{equation*}
            \varepsilon = \inf B - \sup A > 0
        \end{equation*}
        By the assumption, we can find $a\in A$ and $b\in B$, such that $b-a<\inf B - \sup A$. But this implies that
        \begin{equation}
            b < \inf B + a - \sup A.
        \end{equation}
        By the definition of $supreme$ and $infimum$, we have
        \begin{align}
            a \leq \sup A &\Longleftrightarrow  a-\sup A \leq 0\\
            b \geq \inf B &\Longleftrightarrow  b-\inf B \geq 0
        \end{align}
        We combine $(6)$ and $(7)$ by the property of inequlities, we have
        \begin{align*}
            b < \inf B + a - \sup A &\leq \inf B +0  = \inf B\\
        &\Downarrow  \\
            b&<\inf B
        \end{align*}
        But this contradicts to $(8)$. Therefore, $\inf B$ could not be greater than $\sup A$, and we have proved the "$\Longleftarrow $" direction.\\
    \end{proof}
    \begin{Problem}
        Using lower and upper sums, show that the function $t \mapsto t^{2}$ is integrable on $[0, x]$ for all $x>0$ and that $\int_{0}^{x} t^{2} d t=\frac{x^{3}}{3}$.
    \end{Problem}
    \begin{Solution}
        \begin{proof}
            To show that the function $t \mapsto t^{2}$ is integrable on $[0,x]$ for all $x>0$, we need to show that for any $\varepsilon>0$, there exists a partition $P$ of $[0,x]$ such that the upper sum $U(f,P)$ and the lower sum $L(f,P)$ satisfy 
            $$
            U(t^2,P)-L(t^2,P)<\varepsilon
            $$ 
            This is equivalent to showing that 
            $$
            \lim_{n \to \infty} U(t^2,P_n) = \lim_{n \to \infty} L(t^2,P_n)
            $$
            where $(P_n)$ is a sequence of partitions, i.e $P_n = \{t_0,t_1,\ldots,t_n\}$. To find the upper and lower sums, we need to find the maximum and minimum values of $t^2$ on each subinterval in $[0,x]$. Since $\frac{d(t^2)}{dt}=2t>0 \:\text{for}\: \forall t \in [0,x]$, $t^2$ is an increasing function on $[0,x]$. Hence, the maximum value on each subinterval is attained at the right endpoint. So if we divide $[0,x]$ into $n$ equal subintervals of length $\Delta t = x/n$, then for each $i=1,2,\dots,n$, we have 
            \begin{equation}
            M_i=\sup \{t^2: t \in [t_{i-1},t_i]\} = (t_i)^2 = \left(\frac{ix}{n}\right)^2
            \end{equation}
            The minimum value on each subinterval is attained at the left endpoint. So for each $i=1,2,\dots,n$, we have 
            \begin{equation}
            m_i=\inf\{t^2: t \in [t_{i-1},t_i]\} t^2 = (t_{i-1})^2 = \left(\frac{(i-1)x}{n}\right)^2
            \end{equation}
            Now, using $(9)$ and $(10)$, we can compute the upper and lower sums as follows. For the upper sum, we have
            \begin{equation}
                U(t^2,P_n) = \sum_{i=1}^{n} M_i(t_i-t_{i-1})= \sum_{i=1}^{n} \left(\frac{ix}{n}\right)^2 \frac{x}{n} = \frac{x^3}{n^3} \sum_{i=1}^n i^2
            \end{equation}
            and for the lower sum, we have
            \begin{equation}
                L(t^2,P_n) = \sum_{i=1}^{n} m_i (t_i-t_{i-1})  = \sum_{i=1}^{n}\left(\frac{(i-1)x}{n}\right)^2 \frac{x}{n} = \frac{x^3}{n^3} \sum_{i=1}^n (i-1)^2
            \end{equation}
            Using some formulas for sums of squares, we can simplify $(11)$ and $(12)$ as:
            \begin{align}
                U(t^2,P_n) &= \frac{x^3}{n^3} \sum_{i=1}^n i^2  = \frac{x^3}{n^3} \left(\frac{n(n+1)(2n+1)}{6}\right)=\frac{x^3}{6}\left(2+\frac{3}{n}+\frac{1}{n^2}\right)\\
                L(t^2,P_n) &=\frac{x^3}{n^3} \sum_{i=1}^n (i-1)^2 = \frac{x^3}{n^3} \left(\frac{n(n-1)(2n-1)}{6}\right)=\frac{x^3}{6}\left(2-\frac{3}{n}+\frac{1}{n^2}\right)
            \end{align}
            Now we can see that as $n$ increases, both upper and lower sums converge to the same limit:
            \begin{equation*}
                \lim_{n \to \infty} U(t^2,P_n) = \lim_{n \to \infty} L(t^2,P_n) = \frac{x^3}{3}
            \end{equation*}
            Therefore, the function $t \mapsto t^2$ is integrable on $[0,x]$ for all $x>0$. Since the value of the limits of the upper and lower sums are the same as $\frac{x^3}{3}$, we can conclude that the definite integral of $t^2$ on $[0,x]$ is
            \begin{equation*}
                \int_{0}^{x} t^{2} d t=\frac{x^{3}}{3}
            \end{equation*}
        \end{proof}
    \end{Solution}
\end{Solution}
\end{document}